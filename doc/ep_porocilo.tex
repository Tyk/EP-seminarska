% -----------------------------------------------------------------------------
% ########################
% # PREDLOGA ZA POROCILO #
% ########################
%
% @author Iztok Starc
% @date   3. december 2008
%
\documentclass[a4paper,12pt]{report}

% -----------------------------------------------------------------------------
% ####################################################
% # UPORABA PAKETOV - NASTAVITEV JEZIKA in KODIRANJA #
% ####################################################
\usepackage[slovene]{babel}
\usepackage[utf8]{inputenc}
\usepackage{lmodern}
\usepackage[T1]{fontenc}
\usepackage{iwona}

% -----------------------------------------------------------------------------
% ######################################
% # VNOS KLJUCNIH PARAMETROV BESEDILA  #
% ######################################

\newcommand{\naslov}     {Cake Shop}
\newcommand{\prviavtor}  {Aleš Matijevič}
\newcommand{\prviindeks} {63000214}
\newcommand{\drugiavtor} {Igor Jončevski}
\newcommand{\drugiindeks}{63070420}
\newcommand{\tretjiavtor} {Mia Erbus}
\newcommand{\tretjiindeks}{63050184}
\newcommand{\kraj}       {Ljubljana}

% -----------------------------------------------------------------------------
% ###################
% # UPORABA PAKETOV #
% ###################
\usepackage[a4paper,left=25mm,right=25mm,top=20mm,bottom=30mm,includehead]{geometry}

\usepackage{graphicx, epsfig}

\usepackage{fancyhdr}

\usepackage[
colorlinks=true, linkcolor=blue, citecolor=red,
%
pdftitle={\naslov},
pdfauthor={\prviavtor, \drugiavtor},
pdfsubject={Poročilo seminarske naloge pri predmetu Elektronsko Poslovanje},
pdfkeywords={spletna prodajalna, PHP, SSL, MySQL}, a4paper, pagebackref=true, unicode]{hyperref}

% -----------------------------------------------------------------------------
\begin{document}

% -----------------------------------------------------------------------------
% ##################
% # NASLOVNA STRAN #
% ##################
\begin{titlepage}
	\begin{center}
	{UNIVERZA V LJUBLJANI\\[10pt] 
	FAKULTETA ZA RAČUNALNIŠTVO IN INFORMATIKO}

	\vspace{65mm}

	{\Large\textbf{\naslov}}

	\vspace{10mm}

	{\large Poročilo seminarske naloge pri predmetu\\[10pt] Elektronsko poslovanje}

	\vfill
	\vspace{60mm}

\hspace{20mm}
\begin{minipage}[t]{70mm}
	{\bf Študenti}\\
	{\prviavtor} ({\prviindeks})\\ 
	{\drugiavtor} ({\drugiindeks})\\
	{\tretjiavtor} ({\tretjiindeks})
\end{minipage}
%\hfill
\begin{minipage}[t]{50mm}
	{\bf Mentor}\\
	David Jelenc
\end{minipage}
%\hspace{20mm}

	\vspace{40mm}

	{	\kraj, \today}
	\end{center}
\end{titlepage}

% -----------------------------------------------------------------------------
% ##################
% # KAZALO VSEBINE #
% ##################

\tableofcontents
% -----------------------------------------------------------------------------
% ############
% # POVZETEK #
% ############
%\begin{abstract}
%\end{abstract}

% -----------------------------------------------------------------------------
% ##################
% # UVOD DOKUMENTA #
% ##################
\chapter{Uvod}

{\it V uvodu podajte kratko in jedrnato predstavitev teme seminarske naloge, navedite seznam uporabljene tehnologije ter naštejte uporabljene mehanizme za nadzor dostopa.}

% -----------------------------------------------------------------------------
% ###################
% # JEDRO DOKUMENTA #
% ###################

% -----------------------------------------------
\chapter{Primeri uporabe}

{\it Kratek in jedrnat opis vlog uporabnikov ter funkcionalnosti, ki so vsaki vlogi dostopne. Če ste implementirali funkcionalnosti, ki se točkujejo z dodatnimi točkami, jih posebej izpostavite.}

% -----------------------------------------------
\chapter{Uporabniške pristopne točke}

{\it Opis uporabniških pristopnih točk. Tukaj lahko dokumentirate uporabniški vmesnik za ključne primere uporabe, tako da zajamete zaslonske slike vaše aplikacije in jih opremite s komentarji.}

% -----------------------------------------------
\chapter{Podatkovni model}

{\it Slika logičnega podatkovnega modela (denimo iz programa MySQL Workbench) ter kratek opis tabel in obrazložitev netrivialnih atributov.}

% -----------------------------------------------
\chapter{Varnost sistema}

{\it Opis mehanizmov nadzora dostopa in ostalih varnostnih mehanizmov implementacije (SSL/TLS, preverjanje vnosov odjemalca, CAPTCHA, uporaba regularnih izrazov, \dots).}

% -----------------------------------------------
\chapter{Izjava o avtorstvu seminarske naloge}

Spodaj podpisani \textit{\prviavtor}, vpisna številka \textit{\prviindeks}, sem (so)avtor seminarske naloge z naslovom \textit{\naslov}. S svojim podpisom zagotavljam, da sem izdelal ali bil soudeležen pri izdelavi naslednjih sklopov seminarske naloge:
\begin{itemize}
    \item Vzorčni sklop 1
	 \item Vzorčni sklop 2
\end{itemize}

Podpis: {\prviavtor}, l.r.

\newpage

Spodaj podpisana \textit{\drugiavtor}, vpisna številka \textit{\drugiindeks}, sem (so)avtor seminarske naloge z naslovom \textit{\naslov}. S svojim podpisom zagotavljam, da sem izdelal ali bil soudeležen pri izdelavi naslednjih sklopov seminarske naloge:
\begin{itemize}
    \item Vzorčni sklop 1
	 \item Vzorčni sklop 2
\end{itemize}

Podpis: {\drugiavtor}, l.r.

\newpage

Spodaj podpisana \textit{\tretjiavtor}, vpisna številka \textit{\tretjiindeks}, sem (so)avtor seminarske naloge z naslovom \textit{\naslov}. S svojim podpisom zagotavljam, da sem izdelal ali bil soudeležen pri izdelavi naslednjih sklopov seminarske naloge:
\begin{itemize}
    \item Vzorčni sklop 1
	 \item Vzorčni sklop 2
\end{itemize}

Podpis: {\tretjiavtor}, l.r.

% -----------------------------------------------
\chapter{Dodatno vzorčno poglavje}

Besedilo poglavja.

\section{Dodatni vzorčni odsek ena}

Besedilo odseka.

\section{Dodatni vzorčni odsek dva}

Besedilo odseka.

\section{Dodatni vzorčni odsek tri}

Besedilo odseka.

\begin{table}[htb]
 \centering
 \begin{tabular}{c || c}
  \textbf{N} & Vsebina\\ \hline\hline
  1 & Vrstica 1\\        \hline
  2 & Vrstica 2\\        \hline
  ... & ... \\
\end{tabular}
\caption{Tabela vrednosti vzorcev}
\label{tab:1}
\end{table}

Besedilo odseka.

\begin{figure}[htb]
	\centering
	\includegraphics[width=13cm]{img/vzorec.jpg}
	\caption{Slika določenega vzorca}
\label{fig:1}
\end{figure}

Besedilo odseka.

% -----------------------------------------------------------------------------
% #######################
% # ZAKLJUCEK DOKUMENTA #
% #######################
\chapter{Zaključek}

Zaključek.

% -----------------------------------------------------------------------------
% ##############
% # LITERATURA #
% ##############
\begin{thebibliography}{99}
\addtocounter{chapter}{1}
\addcontentsline{toc}{chapter}{\protect\numberline{\thechapter}Literatura}
\addtocontents{toc}{\protect\vspace{15pt}}

\bibitem{bib:ref} Yank K. \emph{Build Your Own Database-Driven Website Using PHP \& MySQL}. SitePoint, 2003. ISBN-10: 0-957-92181-0.

\bibitem{bib:ref1} Michele D.; Jon P. \emph{Learning PHP and MySQL}. O'Rielly, 2006. ISBN-10: 0-596-10110-4.

\bibitem{bib:ref2} Tim C.; Joyce P.; Clark M. \emph{PHP5 and MySQL Bible}. Wiley Publishing, Inc., 2004. ISBN-10: 0-7645-5746-7

\bibitem{bib:LinuxCommandReference} Red Hat Software inc. \emph{Linux Complete Command Reference}. Sams Publishing, 1997. ISBN-10: 0-672-31104-6.

\bibitem{bib:IPsecHowTo1} Ralf Spennberg. \emph{IPsec HOWTO} (online). 2003. (citirano \today). Dostopno na naslovu:
\url{http://www.ipsec-howto.org/t1.html}

\end{thebibliography}

% -----------------------------------------------------------------------------
% ###########
% # DODATEK #
% ###########

 \appendix

\chapter{Naslov dodatka}
{\it Po potrebi.}

\end{document}
